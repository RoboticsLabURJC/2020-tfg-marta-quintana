\chapter{Anexos}
\label{chap:anexos}
\section{Anexo A}

Para probar el funcionamiento de Teachable Machine con nuestro modelo se hizo este ejercicio de reconocimiento de audio sin fusionarlo con Websim. En este enlace \footnote{https://martaquintana.github.io/Audio\_Recognition/index.html} se puede probar el funcionamiento de este código basado en HTML5, CSS3, JavaScript y Teachable Machine(TensorFlowJS).

Este es el código JavaScript importado de Teachable Machine para el reconocimiento de audio: 

\begin{lstlisting}
<html>
<head>
  <script src="https://cdn.jsdelivr.net/npm/@tensorflow/tfjs@1.3.1/dist/tf.min.js"></script>
  <script src="https://cdn.jsdelivr.net/npm/@tensorflow-models/speech-commands@0.4.0/dist/speech-commands.min.js"></script>
  <style type="text/css">
    button{
      text-decoration: none;
      padding: 10px;
      font-weight: 600;
      font-size: 20px;
      color: #ffffff;
      background-color: #AAA;
      border-radius: 6px;
      border: 2px solid black;
    }
    button:hover{
      color: black;
      background-color: #ffffff;
    }
    button:active {
    box-shadow: 0 2px #666;
    transform: translateY(2px);
  }

  </style>
</head>
<body>

<div>Teachable Machine Audio Model</div>
<button type="button" onclick="init()">Start</button>
<h3 id = "prediction" > The most probable word: </h3>
<div  id="label-container"  style="background-color: #D3D3D3;"></div>

<script type="text/javascript">
    // more documentation available at
    // https://github.com/tensorflow/tfjs-models/tree/master/speech-commands

    // the link to your model provided by Teachable Machine export panel
    const URL = "https://teachablemachine.withgoogle.com/models/P3XdF5r5d/";

    async function createModel() {
        const checkpointURL = URL + "model.json"; // model topology
        const metadataURL = URL + "metadata.json"; // model metadata

        const recognizer = speechCommands.create(
            "BROWSER_FFT", // fourier transform type, not useful to change
            undefined, // speech commands vocabulary feature, not useful for your models
            checkpointURL,
            metadataURL);

        // check that model and metadata are loaded via HTTPS requests.
        await recognizer.ensureModelLoaded();

        return recognizer;
    }

    async function init() {
        const recognizer = await createModel();
        const classLabels = recognizer.wordLabels(); // get class labels
        const labelContainer = document.getElementById("label-container");
        const word = document.getElementById("prediction");
        for (let i = 0; i < classLabels.length; i++) {
            labelContainer.appendChild(document.createElement("div"));
        }

        // listen() takes two arguments:
        // 1. A callback function that is invoked anytime a word is recognized.
        // 2. A configuration object with adjustable fields
        recognizer.listen(result => {
            const scores = result.scores; // probability of prediction for each class
            var word_index = 0;
            // render the probability scores per class
            for (let i = 0; i < classLabels.length; i++) {
                const classPrediction = classLabels[i] + ": " + result.scores[i].toFixed(2);
                labelContainer.childNodes[i].innerHTML = classPrediction;
                //The most probable word
                if (result.scores[i].toFixed(2) >= result.scores[word_index].toFixed(2)) {
                   word_index = i;

                }
            }
            var prediction = classLabels[word_index];
            word.innerHTML = "The most probable word: " + prediction;
            //console.log(prediction);
        }, {
            includeSpectrogram: true, // in case listen should return result.spectrogram
            probabilityThreshold: 0.75,
            invokeCallbackOnNoiseAndUnknown: true,
            overlapFactor: 0.50 // probably want between 0.5 and 0.75. More info in README
        });

        // Stop the recognition in 5 seconds.
        // setTimeout(() => recognizer.stopListening(), 5000);
    }
</script>
</<body>
</html>

\end{lstlisting}

