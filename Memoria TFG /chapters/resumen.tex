\newpage
\thispagestyle{plain}			% Supress header 
\setlength{\parskip}{0pt plus 1.0pt}
\section*{Resumen}
Gracias al avance de la tecnología, la robótica y la programación se han convertido en pilares fundamentales para el futuro de los más jóvenes, es por ello que las plataformas web cobran mucha importancia en el aprendizaje. Kibotics es una plataforma web de robótica educativa enfocada a niños y adolescentes en la que pueden aprender los fundamentos de programación de robots en lenguajes como Python y Scratch. Jugar permite desarrollar aspectos psíquicos, físicos y sociales, por estos motivos, el aprendizaje a través de juegos en el entorno educativo y profesional, más conocido como \textit{gamificación}, es fundamental para la formación académica de las nuevas generaciones.
\\
 \\
En este proyecto nos hemos centrado en la realización de tres nuevos ejercicios educativos en formato juego, divertidos y realistas para Kibotics. El primer ejercicio  es un aspirador robótico en el que se ha hecho un nuevo robot que aspira piezas de confeti que están esparcidas por una habitación. El segundo  ejercicio  es el juego del pañuelo, se ha creado un robot Mbot con pinzas que es capaz de coger un lata y recorrer un circuito con ella. 
Estos dos ejercicios poseen evaluadores automáticos que los hacen más interesantes y competitivos. El tercer ejercicio es el teleoperador acústico. En él,  el usuario tiene que guiar con la voz a un dron para que se mueva por el escenario sin chocarse. Esto se ha realizado con procesamiento de audio gracias a la herramienta \textit{Teachable Machine}.
Para una mejor experiencia de usuario, se ha  añadido la posibilidad de poner bandas sonoras a los ejercicios ya existentes y efectos de sonido cuando el robot choca con algún objeto de la escena.
\\
\\
La implementación de los tres ejercicios ha sido principalmente con tecnologías web entre las que cabe destacar JavaScript y el simulador Websim basado en A-Frame.
\\
El ejercicio del aspirador robótico y el juego del pañuelo ya están disponibles en la plataforma para que los usuarios puedan aprender y jugar con ellos. El teleoperador acústico está integrado como prototipo.

