\newpage
\thispagestyle{plain}			% Supress header 
\setlength{\parskip}{0pt plus 1.0pt}
\section*{Resumen}
Este Trabajo de Fin de Grado describe el desarrollo llevado a cabo para la \textit{Gamificación} de la plataforma Kibotics.  
Gracias al avance de las tecnologías, la robótica y la programación se han convertido en pilares fundamentales para el futuro de los más jóvenes, es por ello que las plataformas web cobran mucha importancia en el aprendizaje. Kibotics es una plataforma de robótica educativa enfocada niños y adolescentes en la que pueden aprender los fundamentos de programación de robots tanto en Python como en Scratch. 

Jugar permite desarrollar aspectos psíquicos, físicos y sociales, por eso, el aprendizaje a través de juegos en el entorno educativo y profesional, más conocido como \textit{Gamificación}, es cada vez más importante para las nuevas generaciones.
 
En este proyecto nos hemos centrado en la realización de tres nuevos ejercicios para la plataforma Kibotics que sean divertidos, realistas y que llamen la atención a los más pequeños.

A lo largo del trabajo se explica cómo se han realizado los tres ejercicios.  La implementación ha sido principalmente con tecnologías web entre las que cabe destacar JavaScript y el simulador Websim basado en A-Frame que utiliza Kibotics.

El primero de ellos es el teleoperador acústico. Este ejercicio se ha planteado como un juego en el que el usuario tiene que guiar con la voz a un dron para que se mueva por el escenario sin chocarse. Esto se ha realizado con procesamiento de audio gracias a la herramienta Teachable Machine.
Para una mejor experiencia de los usuarios se han implementado la posibilidad de añadir bandas sonoras a los ejercicios ya existentes y efectos de sonido cuando el robot choca con algún objeto de la escena con eventos de colisión.

El segundo ejercicio es un aspirador robótico en el que se ha hecho un nuevo robot que aspira piezas de confeti que están esparcidas por una habitación. 

Y por último, se ha realizado el ejercicio del juego del pañuelo, es un robot Mbot con pinzas que es capaz de coger un lata y recorrer un circuito con ella. 

Estos dos últimos poseen evaluadores automáticos que hacen que los ejercicios sean más interesantes y competitivos.



