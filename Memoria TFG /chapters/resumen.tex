\newpage
\thispagestyle{plain}			% Supress header 
\setlength{\parskip}{0pt plus 1.0pt}
\section*{Resumen}
Este Trabajo de Fin de Grado describe el desarrollo llevado a cabo para la \textit{Gamificación} de la plataforma Kibotics.  
Gracias al avance de las tecnologías, la robótica y la programación son pilares fundamentales para el futuro de los más jóvenes, es por ello que las plataformas web cobran mucha importancia en el aprendizaje.  Kibotics es una plataforma de robótica educativa enfocada niños y adolescentes en la que pueden aprender los fundamentos de programación de robots tanto en Python como en Scratch. 

Jugar permite desarrollar aspectos psíquicos, físicos y sociales, por eso, el aprendizaje a través de juegos en el entorno educativo y profesional más conocido como \textbf{ \textit{Gamificación} } es cada vez más importante en las nuevas generaciones.
 
En este proyecto nos hemos centrado en la realización de nuevos ejercicios para la plataforma que sean atractivos para llamar la atención de los más pequeños y también se asemejen a robots en los que están más familiarizados.  A lo largo del trabajo se explica como se han desarrollado los tres ejercicios que se han realizado.

 El primero de ellos es el Teleoperador Acústico. Este ejercicio se ha planteado como un juego en el que el usuario tiene que guiar con la voz a un dron para que se mueva por el escenario sin chocarse. Esto se ha realizado con procesamiento de audio gracias a la herramienta Teachable Machine.
Para una mejor experiencia de los usuarios se han implementado la posibilidad de añadir bandas sonoras a los ejercicios ya existentes y  efectos de sonido cuando el robot choca con algún objeto de la escena por eventos de colisión.

El segundo ejercicio es un aspirador robótico coloquialmente conocido como ``Roomba'' en el que se ha hecho un nuevo robot que aspira piezas de confeti que están resparcidas por una habitación. 

Y por último pero no menos importante, se ha realizado el ejercicio del juego del pañuelo, es un robot  Mbot con pinzas que es capaz de coger un lata y recorrer un circuito con ella. 

Estos dos últimos poseen evaluadores automáticos que hacen que los ejercicios sea más interesantes y competitivos.
 
La implementación de los ejercicios ha sido principalmente con tecnologías web en la que cabe destacar JavaScript y el simulador Websim basado en A-Frame que utiliza Kibotics.

% KEYWORDS (MAXIMUM 10 WORDS)
\vfill
Palabras clave: Gamificación, Tecnologías Web, Kibotics, Websim, A-Frame, JavaScript...


