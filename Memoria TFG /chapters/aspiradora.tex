\chapter{Ejercicio aspiadora robótica atrapa confeti}\label{chap:aspiradora}
En este capítulo se explica el desarrollo del ejercicio de una aspiradora robótica que atrapa confeti. Por un lado se expone el enunciado y objetivos del ejercicio, por otra parte cómo se ha implementando en la plataforma Kibotics y finalmente una solución de referencia.


\section{Enunciado}
El objetivo de este ejercicio es hacer una aspiradora robótica que se encuentre en una habitación y sea capaz de recoger piezas de confeti cuando esta pase por encima. Para tener un ejercicio más completo se creará un evaluador automático para contar el confeti recogido en un tiempo determinado.

El alumno deberá programar en Scratch o en Python un algoritmo de planificación de ruta para que la aspiradora consiga atrapar el mayor número de confetis de la habitación en 5 minutos.

\section{Desarrollo del ejercicio}
Para hacer este ejercicio se utilizaron las herramientas Blender, el simulador Websim de Kibotics que contiene la tecnología A-Frame y los lenguajes JavaScript, HTML5 y JSON.

Lo primero que se realizó para crear este ejercicio, fueron varios prototipos en Blender de una aspiradora robótica. En las Figura 5.1 se pueden ver el modelo que se ha creado y  utilizado finalmente para el ejercicio del aspirador robótico.
 
 \begin{figure}[H]
  \begin{subfigure}[b]{0.5\textwidth}
  \centering
    \includegraphics[width=0.95\textwidth, height=0.7\textwidth]{chapters/images/roombablender.png}
    \caption{}
    \label{fig:f1}
  \end{subfigure}
  \hfill
  \begin{subfigure}[b]{0.5\textwidth}
  \centering
    \includegraphics[width=0.95\textwidth, height=0.7\textwidth]{chapters/images/roombablender2.png}
	\caption{}    
    \label{fig:f2}
 
  \end{subfigure}
  \caption{Modelo de la aspiradora robótica  realizado en Blender.}
\end{figure}

A la par se creó el modelo de confeti con la etiqueta \textit{a-cylinder} de A-Frame para hacer pruebas de absorción con código JavaScript.
Las pruebas de absorción se estudiaron tanto en función de la posición como por colisión. En este video \footnote{https://www.youtube.com/watch?v=xkC\_qHXKUDs} se puede ver el primer prototipo con colisión y en este otro  por posición \footnote{https://www.youtube.com/watch?v=If2XMcr1ci4}, al ser un efecto más parecido a lo que ocurre en la realidad se eligió la absorcción basada en posición, ver Figura 5.2.

    
\begin{figure}[H]
  \begin{subfigure}[b]{0.5\textwidth}
  \centering
    \includegraphics[width=0.8\textwidth, height=0.5\textwidth]{chapters/images/prototiporoomba.png}
    \caption{Modelos aspiradora y confeti en A-Frame}
    \label{fig:f1}
  \end{subfigure}
  \hfill
  \begin{subfigure}[b]{0.5\textwidth}
  \centering
    \includegraphics[width=0.8\textwidth, height=0.5\textwidth]{chapters/images/prototiporoomba2.png}
	\caption{Confeti invisible}    
    \label{fig:f2}
 
  \end{subfigure}
  \caption{Absorción por posición, el confeti es invisble cuando el robot pasa por encima.}
\end{figure}

El simulador Websim analiza (\textit{parsea}) ficheros JSON para formar mundos con la tecnología A-Frame. De esta forma es sencillo crear los objetos con pares \textit{``atributo":valor}. A continuación podemos ver como se define el id robot del nuevo robot, se importa el gltf-model que hemos creado en Blender de la aspiradora, y se le asignan otros atributos como la posición, escala y rotación del robot. El dynamic-body nos facilita el movimiento del robot y el atributo \textit{collide} asigna una malla de colisión para que la aspiradora pueda chocarse con los demás elementos del mundo usando físicas de A-Frame y todo sea mucho más realista.

\begin{lstlisting}
   [...]
        {
            "tag": "a-robot",
            "attr": {
                "id": "a-pibot",
                "gltf-model":"/static/websim/assets/models/roombajderbotgrey.gltf",
                "scale": { "x":2, "y":2, "z":2},
                "position": { "x":0, "y":4, "z":30},
                "rotation": { "x":0, "y":90, "z":0},
                "dynamic-body":{"mass": 10},
                [...]
                "collide":{}

            },
           
  [...]
\end{lstlisting}

El confeti no se crea desde el fichero de configuración dado que se tedrían que crear uno a uno y esto extendería demasiado el código del fichero de configuración. Lo que se ha hecho es que una vez esta cargado el mundo en el navegador del usuario, usando JavaScript, creamos dinámicamente todos los confetis.

Para hacerlo de esta forma, primero se generó un programa para fijar aleatoriamente las posiciones de los confetis en el mundo, así los confetis quedan esparcidos por la habitación uniformemente y todos los alumnos cuentan con el mismo escenario. Estas posiciones x, y, z de los confetis se guardaron en otro fichero JSON llamado data.json. En total el escenario está formado por 100 confetis de colores, el color de cada confeti se elige aleatoriamente cuando se crea  desde JavaScript .

Para leer data.json se utiliza:
\begin{lstlisting}

<script type="text/javascript" src=""></script>
\end{lstlisting}
en la plantilla html del ejercicio.

\begin{lstlisting}

function getRandomColor() {
    var letters = '0123456789ABCDEF';
    var color = '#';
    for (var i = 0; i < 6; i++) {
      color += letters[Math.floor(Math.random() * 16)];
    }
    return color;
}

document.addEventListener('robot-loaded', (evt)=>{
    localRobot = evt.detail;
    console.log(localRobot);

    var sceneEl = document.querySelector('a-scene');

    // CREATE CONFETI
    var n = 0;
    var n_confetis = 99;
    score = 0;
    var array = JSON.parse(data);

    for ( n = 0; n <=n_confetis ; n++) {
      var c = document.createElement('a-cylinder');
      var num_conf="confeti"+ String(n)
      c.setAttribute('id', num_conf);
      pos = {x:array[n].x, y:0,z:array[n].z}
    
      c.setAttribute('position',pos);
    
      var color = getRandomColor();
      c.setAttribute('color', color);
      c.setAttribute('height', "0.25");
      c.setAttribute('radius', 1);
      sceneEl.appendChild(c);
}
\end{lstlisting}

En el código anterior podemos ver como se crea el confeti  con  a-cylinder. El id es confeti más un número  n que es un número de 0-99 que se le asigna para crear 100 confetis diferentes (Ejemplo  del id del confeti número 50   id=``confeti50"). Además se le asigna los atributos posición con las posiciones que se leen del data.json, el color aleatorio. Como es un cilindro,además, se define la altura y radio del confeti.
Los confetis no cuentan con malla de colisión para que el aspirador pueda pasar por encima de ellos y absorberlos de una forma más natural que si lo hacemos por colisión.

La absoción se implementó en JavaScript. Se utilizó la función setInterval que ejecuta las funciones que esten dentro de esta función  indefinidamente cada un cierto periodo de tiempo.
Cada 25ms en este programa se comprueba la distancia entre la aspiradora robótica y cada uno de los confetis. La posición del robot es su centro de masas, por eso se  utilizó la distancia euclídea para calcular la distancia entre el centro del robot y el centro del confeti n, si esta distancia d es menor o igual a 2, el confeti n cambia su atributo 'visible' a false. 


Dentro de la absorción se estableció el evaluador automático. Cada vez que d es menor o igual a 2 se suma un punto e indica que el confeti n ha sido  ``absorbido" por el aspirador. La puntuación máxima es de 100, dado que depende del número de confetis 0-99. También se establece una cuenta atrás de 5:00 minutos y cuando llega a 0:00 se deja contar y absorber los confetis. Aunque la aspiradora siga pasando por encima de otros confetis que siguen visibles en el escenario, estos no serán absorbidos ni se aumentará el contador del evaluador.

\begin{lstlisting}

startEvaluator = () => {
  started = true;
}
roomba=sceneEl.querySelector('#a-pibot')

 setInterval(function(){
       //console.log("Roomba",roomba.getAttribute('position').z);
      // console.log("Confeti",confeti.getAttribute('position'));
       //console.log("Confeti",confeti.getAttribute('position').z)
       for ( n = 0; n <=n_confetis ; n++) {
        d = Math.sqrt(Math.pow((array[n].z-roomba.getAttribute('position').z), 2)+Math.pow((array[n].x-roomba.getAttribute('position').x), 2));

       if ( d <= 2 ){
         num_conf="#confeti"+ String(n)
         confeti=sceneEl.querySelector(num_conf)
         if (confeti.getAttribute('visible') == true) {
	      
	      var counter= document.getElementById('time').innerHTML;
		// Tiempo: 00:00	
	      if((counter[8] =='0') && (counter[9]=='0') && (counter[11]=='0') && (counter[12]== '0')){      
		score = score;
		}else{
		score+=1;
		document.getElementById('confeti_recogido').innerHTML = "Confetti recogido: "+ score;
		}
         }
         confeti.setAttribute('visible', false);


        }
      }
	
    }, 25);
 });
\end{lstlisting}


Una vez se consiguió que los confetis desaparecieran del mundo cuando la aspiradora pasara por encima, se mejoró el mundo creando nuevos muebles con Blender, en este caso queríamos que el escenario fuera una habitación de una casa.

\begin{figure}[H]
\centering
\includegraphics[width=0.8\textwidth, height=0.4\textwidth]{chapters/images/habitacionsin.png}
\caption{Habitación sin amueblar}
\end{figure}

\begin{figure}[H]
\centering
\includegraphics[width=0.8\textwidth, height=0.4\textwidth]{chapters/images/habitacioncon.png}
\caption{Habitación amueblada}
\end{figure}

Para que el escenario tenga mayor dificultad se introdujeron 2 pelotas con movimiento. Para conseguir el movimiento de los objetos había dos posibilidades: animación desde Blender o Animación desde A-Frame. 
Se estudiaron ambas opciones. En el vídeo \footnote{https://www.youtube.com/watch?v=1JPb3Mw8638}  se puede ver como hacer una animación no lineal sencilla en Blender para poder usarla en Websim. Se optó por la animación en A-Frame desde el fichero de configuración que era el método ideal para Websim, de esta forma dejamos toda la creación de las pelotas, asignación de texturas, posición y  animación definido  en la configuración.

**Animaciones en Blender y Aframe poner lo de las pelotas**

Finalemente se creó la página web de teoría del ejercicio en HTML5 y se modificaron las plantillas que utiliza el servidor de Kibotics para ver el ejercicio. Este ejercicio se planteó como un juego sencillo en el que se tiene que hacer un algoritmo de planificación de ruta utilizando los sensores y actuadores del robot  para que cuando detecte que un objeto está próximo se mueva en ángulos aleatorios. 
En la teoría se explica el objetivo del ejercicio, los requisitos, un poco de teoría para explicar al alumno los diferentes algoritmos de cobertura que puede utilizar  un aspirador robótico, unas pequeñas pistas para que les ayuden a realizar el ejercicio y un ¿Sabías que...? hablando de las primeras aspiradoras y como funcionaban. 


\begin{figure}[H]
    \centering
    \includegraphics[width=0.8\textwidth, height=0.4\textwidth]{chapters/images/teoria1.png}
    \caption{Página de teoría enunciado y requisitos}
    \label{fig:my_label}
\end{figure}
\begin{figure}[H]
    \centering
    \includegraphics[width=0.8\textwidth, height=0.4\textwidth]{chapters/images/teoria2.png}
    \caption{Teoría}
    \label{fig:my_label}
\end{figure}
\begin{figure}[H]
    \centering
    \includegraphics[width=0.8\textwidth, height=0.4\textwidth]{chapters/images/teoria3.png}
    \caption{Pistas y sabías que.. ?}
    \label{fig:my_label}
\end{figure}



\section{Solución de referencia}

Una de las posibles soluciones de este ejercicio en Scratch es la que se muestra en la Figura 5.8 En la Figura 5.9 se puede ver una solución en Python  para el mismo ejercicio.

\begin{figure}[H]
    \centering
    \includegraphics[width=0.8\textwidth, height=0.4\textwidth]{chapters/images/solucionroombascratch.png}
    \caption{Solución en Scratch }
    \label{fig:my_label}
\end{figure}
\begin{figure}[H]
    \centering
    \includegraphics[width=0.8\textwidth, height=0.4\textwidth]{chapters/images/solucionroombapython.png}
    \caption{Solución en Python}
    \label{fig:my_label}
\end{figure}


Este ejercicio ya está disponible en la plataforma Kibotics. También se han realizado dos videos promocionales para presentar el nuevo ejercicio de la plataforma y como ejemplo para  los alumnos que vayan a realizar el ejercicio. Videos promocionales Python  \footnote{https://www.youtube.com/watch?v=5Q0TmwunYWY} y Scratch \footnote{https://www.youtube.com/watch?v=Twc9wsPFjaY}